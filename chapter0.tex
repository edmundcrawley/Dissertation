%% FRONTMATTER
\begin{frontmatter}

% generate title
\maketitle

\begin{abstract}

These essays study individual consumption behavior and its implications for macroeconomics. Starting with microeconomic measurement, chapter one improves on methods to estimate how households respond to income shocks. Chapter two builds on these methods, applying them to registry data from Denmark, and uses the resulting estimates to calculate the size of redistribution channels of monetary policy. Chapter three models the redistribution channels in a New Keynesian framework.

Chapter one builds on Working's observation that time aggregation of a random walk induces serial correlation in the first differences that is not present in the original series. This important contribution has been overlooked in a large recent literature analyzing income and consumption in panel data. This chapter takes \cite{blundell_consumption_2008} as an example and shows how to correct for this problem. I find the estimate for the partial insurance to transitory shocks, originally estimated to be 5\%, is equal to 24\% when corrected for time aggregation. This estimate is much closer to estimates from the literature that uses natural experiments to estimate the marginal propensity to consume out of transitory shocks.

Chapter two aims to test the microfoundations of consumption models and quantify the macro implications of consumption heterogeneity. We propose a new empirical method to estimate the sensitivity of consumption to permanent and transitory income shocks for different groups of households. We then apply this method to administrative data from Denmark. The large sample size, along with detailed household balance sheet information, allows us to finely divide the population along relevant dimensions. For example, we find that households who stand to lose from an interest rate hike are significantly more sensitive to income shocks than those who stand to gain. Following a one percentage point rate increase, we estimate consumption will decrease by 26 basis points through this interest rate exposure channel alone, making this channel substantially larger than the intertemporal substitution channel that is the key mechanism in representative agent New Keynesian models.

In chapter three we analyze the transmission mechanism of monetary policy to consumption in New Keynesian models with heterogeneous agents. We show that in these models the countercyclical nature of profits, empirically false, plays a large role in amplifying the intertemporal substitution channel. On the other hand the interest rate exposure channel, empirically large, plays a small role. Our analysis makes use of the partial equilibrium decomposition of \cite{auclert_monetary_2017} which we show to perform well even in models where the assumptions do not hold. We suggest expanding the role of the interest rate exposure channel, while dampening the amplification effect of countercyclical profits, is of primary quantitative importance in future work.


\vspace{1cm}

\keywords{Consumption, Marginal Propensity to Consume, Heterogeneity, Monetary Policy, Redistribution}

\JEL{D12, D31, D91, E21}

\Advisors{\\ Professor Christopher Carroll\\ Professor Jon Faust\\ Professor Jonathan Wright}


\end{abstract}

\begin{acknowledgment}

Professor Christopher Carroll has provided invaluable guidance and support throughout my PhD and I feel incredibly lucky to have happened upon him as my intellectual role model and adviser. I am particularly grateful to Professors Jon Faust and Jonathan Wright for their advice and encouragement. These essays have also been significantly improved through conversations with Professors Greg Duffee, Larry Ball, Olivier Jeanne and Bob Barbera. I would also like to thank my coauthors: Andreas Kuchler who battled through Danish registry data with me in chapter two; and Seungcheol Lee who helped me overcome the hurdles of programming HANK models in chapter three. Frederico Ravenna employed me as a visiting scholar at Danmarks Nationalbank, allowing me invaluable access to their data.

Finally these essays would not have been possible without the never ending support of my wife, Christine. She has always shown faith in me, especially when I was lacking it myself, and demonstrated her supermum abilities looking after our daughter while I was away researching in Denmark.
\end{acknowledgment}

\begin{dedication}
 
This thesis is dedicated to Christine.

\end{dedication}

% generate table of contents
%\tableofcontents

% generate list of tables
%\listoftables

% generate list of figures
%\listoffigures

\end{frontmatter}
